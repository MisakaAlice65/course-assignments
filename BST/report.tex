\documentclass[UTF8]{ctexart}
\usepackage{geometry, CJKutf8}
\geometry{margin=1.5cm, vmargin={0pt,1cm}}
\setlength{\topmargin}{-1cm}
\setlength{\paperheight}{29.7cm}
\setlength{\textheight}{25.3cm}

% useful packages.
\usepackage{amsfonts}
\usepackage{amsmath}
\usepackage{amssymb}
\usepackage{amsthm}
\usepackage{enumerate}
\usepackage{graphicx}
\usepackage{multicol}
\usepackage{fancyhdr}
\usepackage{layout}
\usepackage{listings}
\usepackage{float, caption}

\lstset{
    basicstyle=\ttfamily, basewidth=0.5em
}

% some common command
\newcommand{\dif}{\mathrm{d}}
\newcommand{\avg}[1]{\left\langle #1 \right\rangle}
\newcommand{\difFrac}[2]{\frac{\dif #1}{\dif #2}}
\newcommand{\pdfFrac}[2]{\frac{\partial #1}{\partial #2}}
\newcommand{\OFL}{\mathrm{OFL}}
\newcommand{\UFL}{\mathrm{UFL}}
\newcommand{\fl}{\mathrm{fl}}
\newcommand{\op}{\odot}
\newcommand{\Eabs}{E_{\mathrm{abs}}}
\newcommand{\Erel}{E_{\mathrm{rel}}}

\begin{document}

\pagestyle{fancy}
\fancyhead{}
\lhead{董来仪, 3230104635}
\chead{数据结构与算法第五次作业}
\rhead{Oct.31th, 2024}

\section{remove实现阐释}
首先创建函数    detachMin

如果当前节点t为空,返回nullptr

如果当前节点t没有左子树,那么t就是最小节点,新建指针指向t的节点,t =t->right,并返回新建指针

如果当前节点t有左子树,那么最小节点必定在左子树中,递归地在左子树中查找最小节点就行了

然后创建函数        remove

首先递归寻找x所在的节点,找到之后分情况讨论

如果该节点没有子树,那么直接删除该节点

如果该节点只有一个子树,那么直接t = t->right或者t = t->left即可

如果左右子树都有,那么使用detachMin函数返回右子树的最小节点,让该节点的值等于最小节点的值

\section{测试的结果}
创建了一个如图所示的二叉搜索树
                        19
                     
            10                      25

        9       15               21       26

                                    23
先删除了19,输出了

9

10

15

21

23

25

26

此结果与预期一致,因为根节点是19,其左右子树均不为空,而右子树的最小节点是21,所以删除19后,根节点的值为21

然后删除了9和10,9没有子树,删除9后10仅有右子树,所以输出为

10

15

21

23

25

26



15

21

23

25

26

再删除不存在的值100,树的结构没有变化,仍然为

15

21

23

25

26

测试结果一切正常。

我用 valgrind 进行测试,发现没有发生内存泄露。


\end{document}

%%% Local Variables: 
%%% mode: latex
%%% TeX-master: t
%%% End: 
