\documentclass[UTF8]{ctexart}
\usepackage{geometry, CJKutf8}
\geometry{margin=1.5cm, vmargin={0pt,1cm}}
\setlength{\topmargin}{-1cm}
\setlength{\paperheight}{29.7cm}
\setlength{\textheight}{25.3cm}

% useful packages.
\usepackage{amsfonts}
\usepackage{amsmath}
\usepackage{amssymb}
\usepackage{amsthm}
\usepackage{enumerate}
\usepackage{graphicx}
\usepackage{multicol}
\usepackage{fancyhdr}
\usepackage{layout}
\usepackage{listings}
\usepackage{float, caption}

\lstset{
    basicstyle=\ttfamily, basewidth=0.5em
}

% some common command
\newcommand{\dif}{\mathrm{d}}
\newcommand{\avg}[1]{\left\langle #1 \right\rangle}
\newcommand{\difFrac}[2]{\frac{\dif #1}{\dif #2}}
\newcommand{\pdfFrac}[2]{\frac{\partial #1}{\partial #2}}
\newcommand{\OFL}{\mathrm{OFL}}
\newcommand{\UFL}{\mathrm{UFL}}
\newcommand{\fl}{\mathrm{fl}}
\newcommand{\op}{\odot}
\newcommand{\Eabs}{E_{\mathrm{abs}}}
\newcommand{\Erel}{E_{\mathrm{rel}}}

\begin{document}

\pagestyle{fancy}
\fancyhead{}
\lhead{董来仪, 3230104635}
\chead{数据结构与算法第六次作业}
\rhead{Nov.7th, 2024}

\section{remove实现阐释}

1.分析目的

删除树中的某个节点,并保持树的平衡



2.找到目标节点

首先,递归地找到要删除的节点——如果目标节点小于当前节点,递归向左子树查找;如果目标节点大于当前节点,递归向右子树查找.



3.删除目标节点

   叶子节点(没有子节点):如果节点没有左右子树,直接删除该节点,并将其父节点的指针设为 'nullptr'.


   只有一个子节点:如果节点只有一个子节点,则直接用这个子节点替代当前节点,删除该节点.


   有两个子节点:找到右子树的最小节点,并用它来替代当前节点.


   删除节点后更新高度:检查节点左右子树的高度,并将当前节点的高度更新为 '1 + max(左子树高度, 右子树高度)'.



4.恢复平衡:

    1.平衡因子:每个节点的平衡因子是左子树的高度减去右子树的高度.平衡因子的值应该在 '-1' 到 '1' 之间.

    如果某个节点的平衡因子超出了这个范围,则说明树不平衡,需要进行旋转操作.



    2.旋转操作:

    
    平衡因子为 '2',左子树比右子树高,且左子树的左子树比左子树的右子树高,通过右旋转恢复平衡.
    
    平衡因子为 '-2',右子树比左子树高,且右子树的右子树比右子树的左子树高,通过左旋转恢复平衡.

    平衡因子为 '2' 且左子树的右子树比左子树的左子树高,需要先对不平衡的子树进行一次旋转,再对根节点进行一次旋转.
    
    平衡因子为 '-2' 且右子树的左子树比右子树的右子树高,需要先对不平衡的子树进行一次旋转,再对根节点进行一次旋转.





\end{document}

%%% Local Variables: 
%%% mode: latex
%%% TeX-master: t
%%% End: 
