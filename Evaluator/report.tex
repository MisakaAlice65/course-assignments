\documentclass[UTF8]{ctexart}
\usepackage{geometry, CJKutf8}
\geometry{margin=1.5cm, vmargin={0pt,1cm}}
\setlength{\topmargin}{-1cm}
\setlength{\paperheight}{29.7cm}
\setlength{\textheight}{25.3cm}

% useful packages.
\usepackage{amsfonts}
\usepackage{amsmath}
\usepackage{amssymb}
\usepackage{amsthm}
\usepackage{enumerate}
\usepackage{graphicx}
\usepackage{multicol}
\usepackage{fancyhdr}
\usepackage{layout}
\usepackage{listings}
\usepackage{float, caption}

\lstset{
    basicstyle=\ttfamily, basewidth=0.5em
}

% some common command
\newcommand{\dif}{\mathrm{d}}
\newcommand{\avg}[1]{\left\langle #1 \right\rangle}
\newcommand{\difFrac}[2]{\frac{\dif #1}{\dif #2}}
\newcommand{\pdfFrac}[2]{\frac{\partial #1}{\partial #2}}
\newcommand{\OFL}{\mathrm{OFL}}
\newcommand{\UFL}{\mathrm{UFL}}
\newcommand{\fl}{\mathrm{fl}}
\newcommand{\op}{\odot}
\newcommand{\Eabs}{E_{\mathrm{abs}}}
\newcommand{\Erel}{E_{\mathrm{rel}}}

\begin{document}

\pagestyle{fancy}
\fancyhead{}
\lhead{董来仪, 3230104635}
\chead{四则运算表达式求值程序}
\rhead{Nov.6th, 2024}

\section{程序的设计思路}

ATTETION:

我修改了一下负数的输入规则,在输入负数应该使用(),否则会出错.例如输入'-6*3'不对,应该输入'(-6)*3'

使用科学计数法时,e,E前后都要有数字,不能省略1

我先没有考虑负数和科学计数法,只处理非负整数和小数的四则运算.

1. 输入和输出规格

输入:一个字符串,表示中缀表达式,包含数字、加(+)、减(-)、乘(*)、除(/)运算符以及括号( ).

输出:计算表达式的结果,最高保留六位小数.



2. 解析表达式

忽略表达式中的空白字符.

识别表达式中的运算符和操作数(数字).

处理括号内的表达式,确保先计算括号内的表达式.



3. 运算符优先级

优先级规则:乘法和除法优先于加法和减法.

括号:括号内的表达式具有最高优先级.



4. 算法选择

栈:使用两个栈,一个用来存储操作数(数字栈),另一个用于存储运算符(运算符栈).

运算符栈:存储运算符和括号,用于确定运算顺序.

操作数栈:存储操作数,用于执行实际的运算.

正则表达式:用于验证和分割表达式中的数字和运算符.



5. 计算过程

扫描表达式:从左到右扫描表达式.

处理数字:当遇到数字时,将其推入操作数栈.

处理运算符:当遇到运算符时,比较其与运算符栈顶运算符的优先级,并在必要时执行栈中的运算.

处理括号:遇到左括号时,将其推入运算符栈;遇到右括号时,执行运算直至遇到左括号.



6. 完成计算



7. 错误处理

如果遇到非法字符或括号不匹配,返回ILLEGAL.



接着我开始考虑科学计数法的数字的处理.


我的思路是构造一个函数

该函数的作用是能够使用正则表达式,将式子里的科学计数法的数字转换为标准的十进制数字,例如2e2替换成200.

这样子就可以计算科学计数法了

之后是负数的处理

参照上面的思路,我的想法也是处理输入的字符串

我发现(-6)无法计算,但是(0-6)可以计算

所以我认为同样可以使用正则表达式

将(-Num)转换为(0-Num)

这样子就可以计算负数了



\section{测试结果}
如下是部分测试与测试结果分析:

因为操作符不能连续使用,所以以下测试均会返回ILLEGAL

输入3++3,输出ILLEGAL

输入3- -3,输出ILLEGAL

输入3+*9,输出ILLEGAL


括号必须匹配

输入(3+6*2,输出ILLEGAL


运算符不能在式子开头或者结尾

输入+3+1,输出ILLEGAL

输入3+1+,输出ILLEGAL



输入3/0,输出ILLEGAL:Division by zero



测试科学计数法和负数

输入2e2+2,输出202

输入(-2e2)*2,输出-400



测试结果一切正常.

我用 valgrind 进行测试,发现没有发生内存泄露.



\end{document}

%%% Local Variables: 
%%% mode: latex
%%% TeX-master: t
%%% End: 
